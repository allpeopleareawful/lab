\documentclass[a5paper,10pt]{book}
\usepackage[OT1]{fontenc}
\usepackage[utf8]{inputenc}
\usepackage[english, russian]{babel}

\usepackage[left=1.5cm,right=1.5cm,top=2cm,bottom=0.5cm,bindingoffset=0cm]{geometry}
\usepackage{setspace}
\linespread{0.6}
\let\emph\textit
\usepackage[symbol*]{footmisc}
\usepackage{amsmath, amssymb}
\usepackage{wasysym}

\begin{document}
    \markboth{\small{\qquad\textsc{ряды фурье\hspace{3.5cm} \small{[гл. XVII}}}}
    {\small{\textsc{{\S \ 16]}\hspace{1.5cm}линейное функциональное пространство}}}

    \setcounter{page}{360}
    \indentАналогично определению модуля вектора по формуле $(4)$ определяется так называемая \textit{норма} элемента $f(x)$ пространства $\Phi$:
    $$
    ||f||=\sqrt{(f,f)}=\sqrt{\int\limits_a^b [f(x)]^2 \mathrm{d}x}.
    \eqno{(10)}
    $$
    
    \textit{Расстояние между элементами} $f(x)$ и $\varphi(x)$ пространства $\Phi$ аналогично формуле (5) будем называть выражение
    $$
    ||f-\varphi||=\sqrt{\int\limits_a^b [f(x)-\varphi(x)]^2 \mathrm{d}x}.
    \eqno{(11)}
    $$
    
    Выражение (11) расстояния между элементами пространства называется \textit{метрикой} пространства. Оно с точностью до множителя $\sqrt{b-a}$ совпадает со среднеквадратичным уклонением $\delta$ определенным в \S \ 7.
    
    Очевидно, что если $f(x)\equiv\varphi(x)$, т.е. $f(x)$ и $\varphi(x)$ совпадают во всех точках отрезка $[a, b]$, то $||f-\varphi||=0$. Но если $||f-\varphi||=0$, то $f(x)=\varphi(x)$ во всех точках отрезка $[a, b]$, кроме конечного числа точек\footnote[1]{Может и быть бесконечное число точек, где $f(x)\neq\varphi(x)$.}. Но в этом случае также говорят, что элементы пространства $\Phi$ тождественны.
    
    Пространство кусочно монотонных ограниченных функций, в котором определены операции $(7)$, $(8)$ и метрика определяется равенством $(11)$, называется \textit{линейным функциональным пространством с квадратичной метрикой}. Элементы прострнства $\Phi$ называются \textit{точками} пространства или \textit{векторами}.
    Рассмотрим, далее, последовательность функций
    $$
    \varphi_1(x), \varphi_2(x), \ldots, \varphi_k(x), \ldots,
    \eqno{(12)}
    $$
    \noindent принадлежащих пространству $\Phi$
    
    Последовательность функций $(12)$ называется \textit{ортогональной на отрезке} $[a, b]$, если при любых $i, j (i\neq j)$ выполняются равенства
    $$
    (\varphi_i,\varphi_j)=\int\limits_a^b \varphi_i(x)\varphi_j(x) \mathrm{d}x=0
    \eqno{(13)}
    $$
    
    На основании равенств $(I)$ \S \ 1 следует, что, например, система функций
    $$
    1, \cos{x}, \sin{x}, \cos{2x}, \cos{3x}, sin{3x}, \ldots_.
    $$
    \noindent ортогональна на отрезке $[-\pi, \pi]$.
    
    Покажем, далее, что разложение функции в ряд Фурье по ортогональным функциям аналогично разложению вектора по
    \newpage
    \noindent ортогональным векторам. Пусть дан вектор
    $$
    A=A_1e_1+A_2e_2+\ldots+A_ke_k+\ldots+A_ne_n.
    \eqno{(14)}
    $$
    
    \noindent Мы предполагаем, что векторы $e_1,e_2,\ldots,e_n$ ортогональны, т. е. если $i \neq j$, то
    $$
    (e_i,e_j)=0.
    \eqno{(15)}
    $$
    
    Чтобы определить проекцию $A_k$, умножаем скалярно первую и левую часть равенства (14) на вектор $e_k$. На основании свойств $(2)$, $(3)$ получаем:
    $$
    (A,e_k)=A_1(e_1,e_k)+A_2(e_2,e_k)+\ldots+A_k(e_k,e_k)+\ldots+A_n(e_n,e_k).
    $$
    
    \noindent Учитывая $(15)$, получаем
    $$
    (A,e_k)=A_k(e_k,e_k),
    $$
    
    \noindent откуда
    $$
    A_k=\frac{(A,e_k)}{(e_k,e_k)}\qquad(k=1, 2, \ldots, n).
    \eqno{(16)}
    $$
    
    Допустим, далее, что функция $f(x)$ разложена по системе ортогональных функций:
    $$
    f(x)=\sum\limits_{k=1}^\infty a_k\varphi_k(x)
    \eqno{(17)}
    $$
    
    Умножая скалярно обе части равенства $(17)$ на $\varphi_k(x)$ и учитывая равенства $9$ и $13$, получим\footnote[1]{Мы предполагаем, что получающиеся в процессе рассмотрения ряды сходятся и почленное интегрирование законно.}
    $$
    (f, \varphi_k)=a_k(\varphi_k, \varphi_k),
    $$
    
    \noindent откуда
    $$
    a_k=\frac{(f, \varphi_k)}{(\varphi_k, \varphi_k)}=\frac{\int\limits_a^b f(x)\varphi(x)\mathrm{d}x}{\int\limits_a^b [\varphi(x)]^2\mathrm{d}x}.
    \eqno{(18)}
    $$
    
    \noindent Формула $(18)$ аналогична формуле $(16)$.
    \noindent Обозначим, далее
    $$
    s_n=\sum\limits_{k=1}^n a_k\varphi_k(x),
    \eqno{(19)}
    $$
    $$
    \delta=||f-s_n|| \qquad (n=1, 2, \ldots).
    \eqno{(20)}
    $$
    
    \noindent Если
    $$
    \lim\limits_{n \to \infty}\delta_n=0,
    $$
    \noindent то система ортогональных функций $(12)$ является \textit{полной} на отрезке $[a, b]$.
    
    Ряд Фурье $(17)$ сходится к функции $f(x)$ \textit{в среднем}.
    
\end{document}
